
% If the calculus has an acronym, define it.
% (e.g. \newcommand{\LK}{\ensuremath{\mathbf{LK}}\xspace})

\calculusName{Bi-Intuitionistic Linear Logic - shallow nested sequent calculus}   % The name of the calculus
\calculusAcronym{BiILL$_{sn}$}    % The acronym if defined above, or empty otherwise. 
\calculusLogic{Bi-Intuitionistic Linear Logic}  % Specify the logic (e.g. classical, intuitionistic, ...) for which this calculus is intended.
\calculusType{nested sequent calculus}   % Specify the calculus type (e.g. Frege-Hilbert style, tableau, sequent calculus, hypersequent calculus, natural deduction, ...)
\calculusYear{2013}   % The year when the calculus was invented.
\calculusAuthor{Alwen Tiu} % The name(s) of the author(s) of the calculus.


\entryTitle{BiILL-sn}     % Title of the entry (usually coincides with the name of the calculus).
\entryAuthor{Jeremy Dawson \and Rajeev Gor\'e}    % Your name(s). Separate multiple names with "\and".


% If you wish, use tags to give any other information 
% that might be helpful for classifying and grouping this entry:
% e.g. \tag{Two-Sided Sequents}
% e.g. \tag{Multiset Cedents}
% e.g. \tag{List Cedents}
% You are free to invent your own tags. 
% The Encyclopedia's coordinator will take care of 
% merging semantically similar tags in the future.


\maketitle


% If your files are called "MyProofSystem.tex" and "MyProofSystem.bib", 
% then you should write "\begin{entry}{MyProofSystem}" in the line below
\begin{entry}{BiILL-sn}  

% Define here any newcommands you may need:
% e.g. \newcommand{\necessarily}{\Box}
% e.g. \newcommand{\possibly}{\Diamond}


\begin{calculus}

% Add the inference rules of your proof system here.
% The "proof.sty" and "bussproofs.sty" packages are available.
% If you need any other package, please contact the editor (bruno@logic.at)


% \begin{figure}[t]
{\small
Cut and identity: 
$
\qquad
\vcenter{
\infer[id]
{p \seq p}{}
}
\qquad
\vcenter{
\infer[cut]
{\Sc,\Vc \seq \Uc, \Tc}
{\Sc \seq \Uc, A & A, \Vc \seq \Tc}
}
$

Structural rules:
$$
\begin{array}{ccc}
\mbox{$
\infer[\drp_1]
{(\Sc \seq \Tc) \seq \Tc'}
{\Sc \seq \Tc, \Tc'}
$}
&
\mbox{$
\infer[\rp_1]
{\Sc \seq (\Tc \seq \Tc')}
{
\Sc, \Tc \seq \Tc'
}
$}
& 
\mbox{$\infer[gl]
{(\Sc, \Tc \seq \Sc') \seq \Tc'}
{(\Sc \seq \Sc'), \Tc \seq \Tc'}
$} \\ \\
\mbox{$
\infer[drp_2]
{\Sc \seq \Tc, \Tc'}
{(\Sc \seq \Tc) \seq \Tc'}
$}
& 
\mbox{$
\infer[rp_2]
{\Sc, \Tc \seq \Tc'}
{\Sc \seq (\Tc \seq \Tc')}
$}
& 
\mbox{$
\infer[gr]
{\Sc \seq (\Sc' \seq \Tc', \Tc)}
{
 \Sc \seq (\Sc' \seq \Tc'), \Tc
}
$}
\end{array}
$$


Logical rules:
$$
\infer[\punit_l]
{\punit \seq \sunit}
{}
\qquad
\infer[\punit_r]
{\Sc \seq \Tc, \punit}
{\Sc \seq \Tc}
\qquad
\infer[\tunit_l]
{\Sc, \tunit \seq \Tc}
{\Sc \seq \Tc}
\qquad
\infer[\tunit_r]
{\sunit \seq \tunit}{}
$$
$$
\infer[\ltens_l]
{\Sc, A \ltens B \seq \Tc}
{\Sc, A, B \seq \Tc}
\qquad 
\infer[\ltens_r]
{\Sc, \Sc' \seq A \ltens B, \Tc, \Tc'}
{\Sc \seq A, \Tc & \Sc' \seq B, \Tc'}
$$
$$
\infer[\parr_l]
{\Sc, \Sc', A \parr B \seq \Tc, \Tc'}
{\Sc, A \seq \Tc & \Sc', B \seq \Tc'}
\qquad
\infer[\parr_r]
{\Sc \seq A \parr B, \Tc}
{\Sc \seq A, B, \Tc}
$$
$$
\infer[\limp_l]
{\Sc, \Sc', A \limp B \seq \Tc, \Tc'}
{
\Sc \seq A, \Tc
&
\Sc', B \seq \Tc'
}
\qquad
\infer[\limp_r]
{\Sc \seq \Tc, A \limp B}
{\Sc \seq \Tc, (A \seq B)}
$$
$$
\infer[\excl_l]
{\Sc, A \excl B \seq \Tc}
{
 \Sc, (A \seq B) \seq \Tc
}
\qquad
\infer[\excl_r]
{\Sc, \Sc' \seq A \excl B, \Tc, \Tc'}
{
 \Sc \seq A, \Tc & 
 \Sc', B \seq \Tc'
}
$$
}
% \caption{The shallow inference system $\biillsn$.}
% \label{fig:fbills}
% \end{figure}


\end{calculus}

% The following sections ("clarifications", "history", 
% "technicalities") are optional. If you use them, 
% be very concise and objective. Nevertheless, do write full sentences. 
% Try to have at most one paragraph per section, because line breaks 
% do not look nice in a short entry.

\begin{clarifications}
In \cite{Clouston13CSL}, nested sequents are defined as below
where $A_i$ and $B_j$ are formulae:
$$
S~~T ::= S_1,\dots,S_k,A_1,\dots,A_m \seq B_1,\dots, B_n, T_1, \dots,T_l
$$
We use $\Gamma$ and $\Delta$ for multisets of formulae and use
$P$, $Q$, $S$, $T$, $X$, $Y$, etc., for nested
sequents, and $\Sc$, $\Xc$, etc., for
multisets of nested sequents and formulae.
\end{clarifications}

\begin{history}
% ToDo: write here short historical remarks about this proof system,
% especially if they relate to other proof systems. 
% Use "\iref{OtherProofSystem}" to refer to another proof system 
% in the Encyclopedia (where "OtherProofSystem" is its ID). 
% Use "\irefmissing{SuggestedIDForOtherProofSystem}" to refer to 
% another proof system that is not yet available in the encyclopedia.
\end{history}

\begin{technicalities}
% ToDo: write here remarks about soundness, completeness, decidability...
\end{technicalities}


% General Instructions:
% =====================

% The preferred length of an entry is 1 page. 
% Do the best you can to fit your proof system in one page.
%
% If you are finding it hard to fit what you want in one page, remember:
%
%   * Your entry needs to be neither self-contained nor fully understandable
%     (the interested reader may consult the cited full paper for details)
%
%   * If you are describing several proof systems in one entry, 
%     consider splitting your entry.
%
%   * You may reduce the size of your entry by ommitting inference rules
%     that are already described in other entries.
%
%   * Cite parsimoniously (see detailed citation instructions below).
%
% 
% If you do not manage to fit everything in one page, 
% it is acceptable for an entry to have 2 pages.
%
% For aesthetical reasons, it is preferable for an entry to have
% 1 full page or 2 full pages, in order to avoid unused blank space.



% Citation Instructions:
% ======================

% Please cite the original paper where the proof system was defined.
% To do so, you may use the \cite command within 
% one of the optional environments above,
% or use the \nocite command otherwise.

% You may also cite a modern paper or book where the 
% proof system is explained in greater depth or clarity.
% Cite parsimoniously.

% Do not cite related work. Instead, use the "\iref" or "\irefmissing" 
% commands to make an internal reference to another entry, 
% as explained within the "history" environment above.

% You do not need to create the "References" section yourself. 
% This is done automatically.




% Leave an empty line above "\end{entry}".

\end{entry}
