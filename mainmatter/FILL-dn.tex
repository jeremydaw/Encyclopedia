
% If the calculus has an acronym, define it.
% (e.g. \newcommand{\LK}{\ensuremath{\mathbf{LK}}\xspace})

\calculusName{Full Intuitionistic Linear Logic - deep nested sequent calculus}   % The name of the calculus
\calculusAcronym{FILL$_{dn}$}    % The acronym if defined above, or empty otherwise. 
\calculusLogic{Full Intuitionistic Linear Logic}  % Specify the logic (e.g. classical, intuitionistic, ...) for which this calculus is intended.
\calculusType{nested sequent calculus}   % Specify the calculus type (e.g. Frege-Hilbert style, tableau, sequent calculus, hypersequent calculus, natural deduction, ...)
\calculusYear{2013}   % The year when the calculus was invented.
\calculusAuthor{Alwen Tiu} % The name(s) of the author(s) of the calculus.


\entryTitle{FILL-dn}     % Title of the entry (usually coincides with the name of the calculus).
\entryAuthor{Jeremy Dawson \and Rajeev Gor\'e}    % Your name(s). Separate multiple names with "\and".


% If you wish, use tags to give any other information 
% that might be helpful for classifying and grouping this entry:
% e.g. \tag{Two-Sided Sequents}
% e.g. \tag{Multiset Cedents}
% e.g. \tag{List Cedents}
% You are free to invent your own tags. 
% The Encyclopedia's coordinator will take care of 
% merging semantically similar tags in the future.


\maketitle


% If your files are called "MyProofSystem.tex" and "MyProofSystem.bib", 
% then you should write "\begin{entry}{MyProofSystem}" in the line below
\begin{entry}{FILL-dn}  

% Define here any newcommands you may need:
% e.g. \newcommand{\necessarily}{\Box}
% e.g. \newcommand{\possibly}{\Diamond}


\begin{calculus}

% \begin{figure}[t]
{\small
Propagation rules:
$$
\infer[pl_1]
{X[\Sc, A \seq (\Sc' \seq \Tc'), \Tc]}
{
 X[\Sc \seq (A, \Sc' \seq \Tc'), \Tc]
}
\qquad
\infer[pr_1]
{X[(\Sc \seq \Tc), \Sc' \seq A, \Tc']}
{X[(\Sc \seq \Tc, A), \Sc' \seq \Tc']}
$$
$$
\infer[pl_2]
{X[\Sc, (\Sc', A \seq \Tc') \seq \Tc]}
{X[\Sc, A, (\Sc' \seq \Tc') \seq \Tc]}
\qquad
\infer[pr_2]
{X[\Sc \seq \Tc, (\Sc' \seq \Tc', A)]}
{X[\Sc \seq \Tc, A, (\Sc' \seq \Tc')]}
$$

Identity and logical rules: In branching rules, $X[~] \in X_1[~] \merge X_2[~]$,
$\Sc \in \Sc_1 \merge \Sc_2$ and $\Tc \in \Tc_1 \merge \Tc_2$.
$$
\infer[id^d]
{X[\Uc, p \seq p, \Vc]}
{\mbox{$X[~]$, $\Uc$ and $\Vc$ are hollow.}}
\qquad
\infer[\punit_l^d]
{X[\punit, \Uc \seq \Vc]}
{\mbox{$X[~]$, $\Uc$ and $\Vc$ are hollow. }}
\qquad
\infer[\punit_r^d]
{X[\Sc \seq \Tc, \punit]}
{X[\Sc \seq \Tc]}
$$
$$
\infer[\tunit_l^d]
{X[\Sc, \tunit \seq \Tc]}
{X[\Sc \seq \Tc]}
\qquad
\infer[\tunit_r^d]
{X[\Uc \seq \tunit, \Vc]}
{\mbox{$X[~]$, $\Uc$ and $\Vc$ are hollow. }}
$$
$$
\infer[\ltens_l^d]
{X[\Sc, A \ltens B \seq \Tc]}
{X[\Sc, A, B \seq \Tc]}
\qquad
\infer[\ltens_r^d]
{X[\Sc \seq A \ltens B, \Tc]}
{X_1[\Sc_1 \seq A, \Tc_1] & X_2[\Sc_2 \seq B, \Tc_2]}
$$
$$
\infer[\limp_l^d]
{X[\Sc, A \limp B \seq \Tc]}
{
X_1[\Sc_1 \seq A, \Tc_1] & X_2[\Sc_2, B \seq \Tc_2]
}
\qquad
\infer[\limp_r^d]
{X[\Sc \seq \Tc, A \limp B]}
{X[\Sc \seq \Tc, (A \seq B)]}
$$
$$
\infer[\parr_l^d]
{X[\Sc, A \parr B \seq \Tc]}
{
 X_1[\Sc_1, A \seq \Tc_1]
 &
 X_2[\Sc_2, B \seq \Tc_2]
}
\qquad
\infer[\parr_r^d]
{X[\Sc \seq A \parr B, \Tc]}
{X[\Sc \seq A, B, \Tc]}
$$
$$
\infer[\excl_l^d]
{X[\Sc, A \excl B \seq \Tc]}
{X[\Sc, (A \seq B) \seq \Tc]}
\qquad
\infer[\excl_r^d]
{X[\Sc \seq A \excl B, \Tc]}
{X_1[\Sc_1 \seq A, \Tc_1] 
&
 X_2[\Sc_2, B \seq \Tc_2]
}
$$
}
% \caption{The deep inference system $\biilldn$. }
% % \caption{The deep inference system $\biilldn$. In branching rules, $X[~] \in X_1[~] \merge X_2[~]$,
% % $\Sc \in \Sc_1 \merge \Sc_2$ and $\Tc \in \Tc_1 \merge \Tc_2$.}
% \label{fig:fbilld}
% \end{figure}

\end{calculus}

% The following sections ("clarifications", "history", 
% "technicalities") are optional. If you use them, 
% be very concise and objective. Nevertheless, do write full sentences. 
% Try to have at most one paragraph per section, because line breaks 
% do not look nice in a short entry.

\begin{clarifications}
In \cite{Clouston13CSL}, nested sequents are defined as below
where $A_i$ and $B_j$ are formulae:
$$
S~~T ::= S_1,\dots,S_k,A_1,\dots,A_m \seq B_1,\dots, B_n, T_1, \dots,T_l
$$
We use $\Gamma$ and $\Delta$ for multisets of formulae and use
$P$, $Q$, $S$, $T$, $X$, $Y$, etc., for nested
sequents, and $\Sc$, $\Xc$, etc., for
multisets of nested sequents and formulae.

Inference rules in $\biilldn$ are applied in a {\em context}, i.e., a
nested sequent with a hole $[~]$. Notice that structural rules are missing
from $\biilldn$. The branching rules require operations to merge contexts
and nested sequents, which are explained below. The zero-premise rules
require that certain sequents or contexts are {\em hollow}, i.e., containing
no occurrences of formulae.

The {\em merge set} $X_1 \merge X_2$ of two sequents $X_1$ and $X_2$
is defined as:
\begin{center}
\begin{tabular}[c]{l@{\extracolsep{1cm}}cl}
% $X_1 \merge X_2 = $
 \multicolumn{3}{l}{
  $X_1 \merge X_2 = \{~ 
   (\Gamma_1, \Gamma_2, Y_1, \dots, Y_m \seq \Delta_1, \Delta_2, Z_1,
    \dots, Z_n) ~\mid$
 }
\\
  & & $X_1 = (\Gamma_1, P_1, \dots, P_m \seq \Delta_1, Q_1,\dots,Q_n)$
  \mbox{ and }
\\
  & & 
  $X_2 = (\Gamma_2, S_1, \dots, S_m \seq \Delta_2, T_1, \dots, T_n)$
 \mbox{ and }
\\
 & &
  $Y_i \in P_i \merge S_i$ for $1 \leq i \leq m$ 
  and $Z_j \in Q_j \merge T_j$ for $1 \leq j \leq n$
  $~\}$
\end{tabular}
\end{center}
When
$X \in X_1 \merge X_2$, we say that $X_1$ and $X_2$ are a {\em
  partition} of $X.$ 

The merge set $X_1[~] \merge X_2[~]$ of 
two contexts $X_1[~]$ and $X_2[~]$ is defined in \cite{Dawson14TCS},
Figure~\ref{fig:merge}.
If $X[~] = X_1[~] \merge X_2[~]$ we say $X_1[~]$ and $X_2[~]$
are a {\em partition} of $X[~]$.
We extend the notion of a merge set between multisets of formulae and sequents
as follows. Given 
$
\Xc = \Gamma \cup \{X_1,\dots,X_n\}
$
and
$
\Yc = \Delta \cup \{Y_1,\dots,Y_n\} 
$
their merge set contains all multisets of the form:
$
\Gamma \cup \Delta \cup \{Z_1,\dots,Z_n\}
$
where $Z_i \in X_i \merge Y_i.$


\end{clarifications}

\begin{history}
% ToDo: write here short historical remarks about this proof system,
% especially if they relate to other proof systems. 
% Use "\iref{OtherProofSystem}" to refer to another proof system 
% in the Encyclopedia (where "OtherProofSystem" is its ID). 
% Use "\irefmissing{SuggestedIDForOtherProofSystem}" to refer to 
% another proof system that is not yet available in the encyclopedia.
\end{history}

\begin{technicalities}
The calculus shown is for Bi-Intuitionistic Linear Logic.
A nested sequent is a (nested) \emph{FILL-sequent} if it has no nesting of
sequents on the left of $\seq$, and no occurrences of $\excl$ at all.

The correspondence of this calculus to the categorical semantics is via
the shallow nested sequent calculus \iref{BiILL-sn} for BiILL; 
only in the deep nested sequent calculus is it obvious that a derivation of a
FILL-sequent uses only steps within FILL.
See \cite{Dawson14TCS} and \cite{Clouston13CSL} for details.
\end{technicalities}


% General Instructions:
% =====================

% The preferred length of an entry is 1 page. 
% Do the best you can to fit your proof system in one page.
%
% If you are finding it hard to fit what you want in one page, remember:
%
%   * Your entry needs to be neither self-contained nor fully understandable
%     (the interested reader may consult the cited full paper for details)
%
%   * If you are describing several proof systems in one entry, 
%     consider splitting your entry.
%
%   * You may reduce the size of your entry by ommitting inference rules
%     that are already described in other entries.
%
%   * Cite parsimoniously (see detailed citation instructions below).
%
% 
% If you do not manage to fit everything in one page, 
% it is acceptable for an entry to have 2 pages.
%
% For aesthetical reasons, it is preferable for an entry to have
% 1 full page or 2 full pages, in order to avoid unused blank space.



% Citation Instructions:
% ======================

% Please cite the original paper where the proof system was defined.
% To do so, you may use the \cite command within 
% one of the optional environments above,
% or use the \nocite command otherwise.

% You may also cite a modern paper or book where the 
% proof system is explained in greater depth or clarity.
% Cite parsimoniously.

% Do not cite related work. Instead, use the "\iref" or "\irefmissing" 
% commands to make an internal reference to another entry, 
% as explained within the "history" environment above.

% You do not need to create the "References" section yourself. 
% This is done automatically.




% Leave an empty line above "\end{entry}".

\end{entry}
